\PassOptionsToPackage{unicode=true}{hyperref} % options for packages loaded elsewhere
\PassOptionsToPackage{hyphens}{url}
%
\documentclass[]{article}
\usepackage{lmodern}
\usepackage{amssymb,amsmath}
\usepackage{ifxetex,ifluatex}
\usepackage{fixltx2e} % provides \textsubscript
\ifnum 0\ifxetex 1\fi\ifluatex 1\fi=0 % if pdftex
  \usepackage[T1]{fontenc}
  \usepackage[utf8]{inputenc}
  \usepackage{textcomp} % provides euro and other symbols
\else % if luatex or xelatex
  \usepackage{unicode-math}
  \defaultfontfeatures{Ligatures=TeX,Scale=MatchLowercase}
\fi
% use upquote if available, for straight quotes in verbatim environments
\IfFileExists{upquote.sty}{\usepackage{upquote}}{}
% use microtype if available
\IfFileExists{microtype.sty}{%
\usepackage[]{microtype}
\UseMicrotypeSet[protrusion]{basicmath} % disable protrusion for tt fonts
}{}
\IfFileExists{parskip.sty}{%
\usepackage{parskip}
}{% else
\setlength{\parindent}{0pt}
\setlength{\parskip}{6pt plus 2pt minus 1pt}
}
\usepackage{hyperref}
\hypersetup{
            pdftitle={tutorial},
            pdfauthor={jjz},
            pdfborder={0 0 0},
            breaklinks=true}
\urlstyle{same}  % don't use monospace font for urls
\usepackage[margin=1in]{geometry}
\usepackage{graphicx,grffile}
\makeatletter
\def\maxwidth{\ifdim\Gin@nat@width>\linewidth\linewidth\else\Gin@nat@width\fi}
\def\maxheight{\ifdim\Gin@nat@height>\textheight\textheight\else\Gin@nat@height\fi}
\makeatother
% Scale images if necessary, so that they will not overflow the page
% margins by default, and it is still possible to overwrite the defaults
% using explicit options in \includegraphics[width, height, ...]{}
\setkeys{Gin}{width=\maxwidth,height=\maxheight,keepaspectratio}
\setlength{\emergencystretch}{3em}  % prevent overfull lines
\providecommand{\tightlist}{%
  \setlength{\itemsep}{0pt}\setlength{\parskip}{0pt}}
\setcounter{secnumdepth}{0}
% Redefines (sub)paragraphs to behave more like sections
\ifx\paragraph\undefined\else
\let\oldparagraph\paragraph
\renewcommand{\paragraph}[1]{\oldparagraph{#1}\mbox{}}
\fi
\ifx\subparagraph\undefined\else
\let\oldsubparagraph\subparagraph
\renewcommand{\subparagraph}[1]{\oldsubparagraph{#1}\mbox{}}
\fi

% set default figure placement to htbp
\makeatletter
\def\fps@figure{htbp}
\makeatother

\usepackage{etoolbox}
\makeatletter
\providecommand{\subtitle}[1]{% add subtitle to \maketitle
  \apptocmd{\@title}{\par {\large #1 \par}}{}{}
}
\makeatother
% https://github.com/rstudio/rmarkdown/issues/337
\let\rmarkdownfootnote\footnote%
\def\footnote{\protect\rmarkdownfootnote}

% https://github.com/rstudio/rmarkdown/pull/252
\usepackage{titling}
\setlength{\droptitle}{-2em}

\pretitle{\vspace{\droptitle}\centering\huge}
\posttitle{\par}

\preauthor{\centering\large\emph}
\postauthor{\par}

\predate{\centering\large\emph}
\postdate{\par}

\title{tutorial}
\author{jjz}
\date{2020/9/15}

\begin{document}
\maketitle

\hypertarget{step-1}{%
\subsubsection{Step 1}\label{step-1}}

\hypertarget{please-select-sequences-preprocessing-on-the-top-bar.}{%
\subparagraph{\texorpdfstring{Please Select \textbf{Sequences
preprocessing} on the top
bar.}{Please Select Sequences preprocessing on the top bar.}}\label{please-select-sequences-preprocessing-on-the-top-bar.}}

\begin{center}\rule{0.5\linewidth}{\linethickness}\end{center}

\hypertarget{step-2}{%
\subsubsection{Step 2}\label{step-2}}

\hypertarget{select-1.-sequences-summary}{%
\subparagraph{\texorpdfstring{Select \textbf{1. Sequences
summary}}{Select 1. Sequences summary}}\label{select-1.-sequences-summary}}

In this step, we will summarize the sequences.

\begin{center}\rule{0.5\linewidth}{\linethickness}\end{center}

\hypertarget{step-3}{%
\subsubsection{Step 3}\label{step-3}}

\hypertarget{select-the-directory-containing-the-sequences-files-and-select-the-sequences-type-single-end-or-paired-end.}{%
\subparagraph{Select the directory containing the sequences files and
select the sequences type (single end or paired
end).}\label{select-the-directory-containing-the-sequences-files-and-select-the-sequences-type-single-end-or-paired-end.}}

\begin{center}\rule{0.5\linewidth}{\linethickness}\end{center}

\hypertarget{step-4}{%
\subsubsection{Step 4}\label{step-4}}

\hypertarget{click-on-the-button-start}{%
\subparagraph{\texorpdfstring{Click on the button
\textbf{``Start!''}}{Click on the button ``Start!''}}\label{click-on-the-button-start}}

\begin{center}\rule{0.5\linewidth}{\linethickness}\end{center}

\hypertarget{step-5}{%
\subsubsection{Step 5}\label{step-5}}

\hypertarget{click-on-the-button-view-to-inspect-the-result.}{%
\subparagraph{\texorpdfstring{Click on the button \textbf{``View!''} to
inspect the
result.}{Click on the button ``View!'' to inspect the result.}}\label{click-on-the-button-view-to-inspect-the-result.}}

\hypertarget{if-you-want-to-know-what-the-result-looks-like-you-can-click-on-the-button-example-for-single-end-or-example-for-paired-end.}{%
\subparagraph{\texorpdfstring{If you want to know what the result looks
like, you can click on the button \textbf{``Example for single end''} or
\textbf{``Example for paired
end''}.}{If you want to know what the result looks like, you can click on the button ``Example for single end'' or ``Example for paired end''.}}\label{if-you-want-to-know-what-the-result-looks-like-you-can-click-on-the-button-example-for-single-end-or-example-for-paired-end.}}

\begin{center}\rule{0.5\linewidth}{\linethickness}\end{center}

\end{document}
